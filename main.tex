\documentclass{beamer}
%
% Choose how your presentation looks.
%
% For more themes, color themes and font themes, see:
% http://deic.uab.es/~iblanes/beamer_gallery/index_by_theme.html
%
\mode<presentation>
{
  \usetheme{default}      % or try Darmstadt, Madrid, Warsaw, ...
  \usecolortheme{default} % or try albatross, beaver, crane, ...
  \usefonttheme{default}  % or try serif, structurebold, ...
  \setbeamertemplate{navigation symbols}{}
  \setbeamertemplate{caption}[numbered]
  \setbeamertemplate{footline}[frame number]
} 

\usepackage[english]{babel}
\usepackage[utf8x]{inputenc}

\title{Functional ntuple filters for ROOT}
\author{Jim Pivarski}
% \institute{Where You're From}
% \date{Date of Presentation}

\begin{document}

\begin{frame}
  \titlepage
\end{frame}

\begin{frame}[fragile]{Motivation}

\begin{block}{}
The process of cutting and transforming ntuples, putting results in plots, is the essence of exploratory data analysis. It's the primary way physicists interact with data and learn things from it.
\end{block}

\begin{block}{}
ROOT has a mini-language for cuts and transformations:
{\scriptsize \begin{verbatim}
ttree->Draw("fNtrack/fNvertex", "fIsValid && fFlag")
\end{verbatim}}
asdf
\end{block}

\end{frame}

\end{document}
